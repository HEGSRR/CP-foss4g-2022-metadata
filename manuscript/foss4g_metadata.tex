% Version 2020-12-15
% update – 161114 by Ken Arroyo Ohori: made spacing closer to Word template throughout, put proper quotes everywhere, removed spacing that could cause labels to be wrong, added non-breaking and inter-sentence spacing where applicable, removed explicit newlines
% update – 010819 by Dennis Wittich: made spacing and font size closer to Word template, updated references and refernces style
% update – 042319 by Dennis Wittich: font size of captions set to 'small', first author names are shortened, hyphenation fixed
% update – 010620 by Dennis Wittich: Footnotes alignment set to left
% update - 151220 by Clement Mallet: Template adapted for double blind full paper submissions
% update - 060321 by Christian Heipke: Template refined for double blind full paper submissions
% update - 090921 by Christian Heipke: Template refined for double blind full paper submissions

\documentclass{isprs} % isprs class modified 23-04-2019 (Dennis Wittich)
\usepackage{subfigure}
\usepackage{setspace}
\usepackage{geometry} % added 27-02-2014 Markus Englich
\usepackage{epstopdf}
\usepackage[labelsep=period]{caption}  % added 14-04-2016 Markus Englich - Recommendation by Sebastian Brocks
\usepackage[british]{babel} 
\usepackage[hang]{footmisc}
\usepackage[authoryear,round,longnamesfirst]{natbib}
\def\footnotemargin{1em} % added 08-01-2020 Dennis Wittich

%\usepackage[authoryear]{natbib}
%\def\bibhang{0pt}

\geometry{a4paper, top=25mm, left=20mm, right=20mm, bottom=25mm, headsep=10mm, footskip=12mm} % added 27-02-2014 Markus Englich
%\usepackage{enumitem}

%\usepackage{isprs}
%\usepackage[perpage,para,symbol*]{footmisc}

%\renewcommand*{\thefootnote}{\fnsymbol{footnote}}
\captionsetup{justification=centering,font=normal} % thanks to Niclas Borlin 05-05-2016
\captionsetup[figure]{font=small} % added 23-04-2019 Dennis Wittich
\captionsetup[table]{font=small} % added 23-04-2019 Dennis Wittich

\begin{document}

\title{Mainstreaming metadata into research workflows to advance reproducibility and open geographic information science}
\date{}


% KAO: Remove extra spacing
% Anonymous submissions, authors' names should not be visible
\author{Joseph Holler\textsuperscript{1} and Peter Kedron\textsuperscript{2}}

% KAO: Remove extra newline
% Anonymous submissions, authors' affiliations should not be visible
\address{\textsuperscript{1}Department of Geography, Middlebury College - josephh@middlebury.edu \\
\textsuperscript{2}School of Geographical Sciences and Urban Planning, Arizona State University - peter.kedron@asu.edu}


% If the corresponding author is NOT the final author, always add a % space before the subsequent comma, i.e.
% first author name\textsuperscript{a,}\thanks{Corresponding author} , % second author name \textsuperscript{b}, etc.
% thanks to Niclas Borlin 05-05-2016


\commission{}{} %This field is optional. If filled, XX and YY should be replaced by adequate numbers. See https://www2.isprs.org/commissions/
\workinggroup{} %This field is optional.
\icwg{}   %This field is optional.

% KAO: Use times symbol
\abstract{
100 - 200 words here  
}

\keywords{Metadata, Open Science, Reproducibility}

\maketitle

%\saythanks % added 28-02-2014 Markus Englich

% KAO: Sloppy spacing ensures non-overfull lines. Can be removed if this is not an issue.
\sloppy

\section{Introduction}\label{Introduction}

Researchers working with geospatial data commonly find themselves spending less time and effort \textit{doing research}---observing, recording, and analyzing data---than they do reading and writing about their data.
Open science principles increase the demands on researchers to create and maintain metadata in support of reproducibility, replicability, research synthesis, and meta-analyses \citep{NASEM2019}.
At the same time, the scientific community is lacking easy-to-use tools for annotating metadata \citep{NASEM2018}.

Prior to analysis, researchers using secondary data must scour metadata documentation and technical manuals for information about their data.
Researchers collecting primary data must painstakingly document data collection and human subjects protocols.
Researchers pre-registering their studies must anticipate and record each procedure for data collection and analysis. During analysis, researchers should record data provenance with sufficient detail to communicate a complete data quality report and reproducible methodology.
After analysis, researchers are increasingly required to archive data and results in a findable, accessible, interoperable, and reusable (FAIR) manner \citep{Wilkinson2016}.

In this research paper, we assert that researchers should formally create and update metadata starting from the inception of a research project and that open source geospatial software can---and should---support metadata-rich research workflows.
Rather than repeatedly writing about their data through all phases of a research workflow, researchers could formalize metadata at the project outset, use the metadata to generate much of the documentation required of the research process, and use software to maintain and version-track metadata throughout the research process.
In the next section, we review the International Standards Organization (ISO) standards for geospatial metadata and Dublin Core standards for research artifacts. We then review the roles and requirements of research documentation in open and reproducible science and in ethical research with human subjects, illustrated with examples from our own reproduction studies.
With metadata requirements for supporting geospatial research in mind, we assess the capabilities of open source geospatial analysis and metadata software to meet those requirements. We conclude with our recommendations for software infrastructure in support of metadata-rich geospatial research workflows.

Free and open source software for geospatial analysis (FOSS4G) supports burgeoning possibilities for practicing open and computationally reproducible human-environment and geographical research \citet{Singleton2016}.
Open and reproducible research practices may accelerate the pace of scientific discovery and enhance the scientific community's functions of knowledge verification, correction, and diffusion \citep{Rey2009,Kedron2022}.
Geospatial metadata provides the foundation for reproducibility and open science and accordingly, requires more support in open source geospatial software.

\section{Geospatial Metadata for Reproducible Open Science}\label{sec:Background}

\subsection{Open Science Workflow}\label{sec:Workflow}

The 2018 \citeauthor{NASEM2018} report \textit{Open Science by Design} envisions open science practices in all six phases of the research workflow: provocation, ideation, knowledge generation, validation, dissemination, and preservation.
Researchers would benefit from open science in the \textbf{provocation} phase of reviewing literature and data through the enhanced findability and accessibility of publications and associated data and code.
In the \textbf{ideation} phase, researchers investigate data, and plan, prototype, and preregister research designs.
In the \textbf{knowledge generation} phase, researchers would use open source software tools to collect and analyze data in interoperable formats with sufficient documentation, metadata, and computational notebooks to enable future reuse and replication. 
In the \textbf{validation} phase, researchers analyze, visualize, interpret, and validate results while sharing preliminary findings in working papers and conferences.
In the \textbf{dissemination} phase, the research peer reviewed, revised, and published, ideally with associated data and code.
Finally, in the \textbf{preservation} phase, the manuscripts, data, and code are placed in FAIR digital archives with final revisions of metadata.

\subsection{Geospatial Metadata}\label{sec:Metadata}

Many disciplines lacking sufficient standards for data sharing, interoperability, and documentation \citep{NASEM2019}.
However, the geographic sciences have the benefit of leaning on the efforts of individual states and regions' spatial data infrastructures (SDIs), including the Federal Geographic Data Committee's (FGDC) Content Standard for Digital Geospatial Metadata (CSDGM) in the United States, and the Infrastructure for Spatial Information in Europe (INSPIRE).
Individual SDIs are increasingly following and harmonizing with the International Standards Organization \citep{ISO2014} series of geographic information metadata standards, starting with standard 19115-1. 
The geographic sciences also have standard formats, protocols, and algorithms for storing, distributing, and analyzing geographic data---all coordinated by the Open Geospatial Consortium (OGC).


ISO

INSPIRE (Infrastructure for Spatial Information in Europe) uses standards EN ISO 19115, EN ISO 19119, and ISO 15836 (Dublin
Core). INSPIRE adds constraints to ISO 19115, e.g. requiring spatial and temporal extents and lineage statements, and at least one keyword must belong to the GEMET thesaurus (https://www.eionet.europa.eu/gemet). INSPRE required spatial metadata as of Article 5(1) of INSPIRE Directive 2007/2/EC.

Dublin Core

\subsection{Open Science and Reproducibility}\label{sec:Metadata}

FAIR, Reproducible, Metaanalyses, Preanalysis Registration

Following the five star guide for reproducibility \citep{Wilson2021}, researchers can achieve four stars by conducting research with open data and software and documenting metadata according to the standards of the International Organization for Standardization (ISO) and OGC (Open Geospatial Consortium).
Metadata is the key to documenting the provenance of research data artifacts, preserving information about every detail of data creation and transformation \citep{Tullis2021}. The FAIR Guiding Principles for scientific data management enumerate functions for metadata in each of the principles for research: findable, accessible, interoperable, and reusable \citep{Wilkinson2016}.
However, open source geospatial software platforms generally lack the tools necessary for mainstreaming geospatial metadata into the full research workflow in support of more efficient research work and enhanced reproducibility and open science.
This research on metadata is part of a larger human-environment and geographical sciences reproducibility and replicability (github.com/HEGSRR) project aimed at conducting formal reproduction and replication studies in the geographical sciences and integrating reproducibility into undergraduate and graduate level curricula in research methods.

Following the National Academies of Science, Engineering and Medicine \citep{NASEM2019}, a reproduction study aims find the same results using the same data and methodology as a published study.
A replication study aims to test the findings of a published study by collecting new data and following a similar methodology, which may intentionally modify one or more research parameters.
Together, reproduction and replication studies offer a deep understanding of the original research, test its credibility and generalizability, and enhance the self-corrective mechanisms of the scientific community.
Metadata is information about data, including essential contextual information about the data's spatial structure, attributes, creation, maintenance, access, licensing, and provenance.

A key component of reproducible research is an executable research compendium containing all of the data, code, and narrative required to compile a research publication from raw data \citep{Nust2021}.
Computational notebooks like Jupyter notebooks or R Markdown are commonly used to interweave narrative with code in executable compendia.
In order to maximize replicability and inferential power, the research compendium should begin with a pre-registered research plan prior to data collection, requiring researchers to fully specify metadata for all of the research data and analyzes that they intend to create \citep{Nosek2018}.

It is recommended to store compendiums in version tracking systems like Git in order to preserve a full history of changes to the research project.
Finally, the compendium should be published parallel to academic publications so that other researchers can independently re-run, check and verify the analysis, or incorporate the research in future projects.
In order to maximize the findability and legibility of the research compendium for both humans and machines, the overall repository and each of its components must be meticulously documented with metadata according to international standards \citep{Wilkinson2016,Wilson2021}.

In this three-part research paper, we focus on metadata in research compendia and related research products through all phases of the research workflow.
First, we specify ideal requirements of geospatial metadata in support of reproducible research workflows and open science.
We consider metadata needs at each step of the research process, including proposal writing, pre-analysis registration, ethics review board approval, data collection and analysis, publishing, and reproducing published research.
The metadata software needs assessment is based on literature review of reproducibility and open science, and on teaching and practicing reproducibility with geographic methods.

Second, we review the Dublin Core and International Organization for Standardization (ISO) geospatial metadata standards and popular open source platforms for geospatial research and their support for the requirements of geospatial metadata articulated in the first part.
The scope of the review includes metadata functionality available through spatial analysis software platforms, including R, Python, QGIS, GRASS and SAGA; and it also includes metadata or cataloging tools, including GeoNetwork, GeoNode, the USGS Metadata Wizard, and mdeditor.

Finally, we articulate a vision for open source geospatial metadata software development in support of open and reproducible human-environment and geographical research.
In this vision, metadata software tools shall integrate with executable research compendium to assist researchers with their workflow from inception to publishing and archiving.
The vision builds off our HEGSRR project, in which we independently reproduce and replicate published studies with open source geospatial software, integrate reproduction and replication studies into project-based geographic information science courses, and develop curricula and infrastructure for reproducible research.
Each section of the paper is thus supplemented with experiences and examples drawn from the HEGSRR project.
Of particular relevance, we have already completed seven reproduction or replication studies with graduate and undergraduate students using open source geospatial software, encountering numerous barriers caused by inadequate use or documentation of metadata in research planning, execution, and archiving.
We have also developed a template Git repository compendium for reproducible research and prototyped its use in our studies and teaching, discovering software barriers to documenting metadata and opportunities to integrate metadata into more efficient and reproducible research practices.

\subsection{What do we need from metadata?}\label{metadataneeds}

Here, we shall first try to list the requirements of our metadata software:

%\itemize
\begin{enumerate}
\setlength\itemsep{0em}\setlength\parskip{0em}\setlength\topsep{0em}\setlength\partopsep{0em}\setlength\parsep{0em} 
\item{Re-use common entities, e.g. the metadata author and spatial support} 
\item{Scan compendium for geospatial data}
\item{Automatically extract information from geospatial data and web services}
\item{Record metadata with ISO standards}
\item{Integration with Git version tracking}
\item{Quality and completeness checks}
\item{API / Application Programming Interface}
\end{enumerate}

Secondly, we shall list the required metadata fields:

%\itemize
\begin{enumerate}
\setlength\itemsep{0em}\setlength\parskip{0em}\setlength\topsep{0em}\setlength\partopsep{0em}\setlength\parsep{0em} 
\item{Title of the dataset} 
\item{Abstract (description)}
\item{Theme Keywords, including ISO19115 Topic}
\item{Place Keywords, look up more on this}
\itme{Purpose (context of data creation)}
\item{Data author\textbf{anonymous} }
\item{Association with larger works}
\item{Publication date}
\item{Link / identifier, e.g. DOI}
\item{Access constraints}
\item{Use constraints}
\item{Dataset credits}
\item{Time period represented in dataset}
\item{Native data set environment (processing environment, file name, file size)}
\item{Data Quality: Attribute accuracy, Logical Consistency, Completeness, Horizontal and Vertical Positional Accuracy}
\item{Process Steps: method, date, inputs, and outputs}
\item{Source inputs, including fairly complete reference information for each data source}
\item{Spatial support: description, bounding box, coordinate reference system, vertical datum, spatial data type, feature count}
\item{Entity and attribute: this is the data dictionary}
\item{Distribution: contact information, access instructions, liability disclaimer, data fees}
\item{Metadata contact}
\end{enumerate}

\begin{table*}[h]
	\centering
		\begin{tabular}{|c|c|c|c|c|}\hline
		    Requirement&QGIS&GeoNode&GeoNetwork&abundant Content\\hline
			 Top&25&1.0&no\\
			 Bottom&25&1.0&yes\\
			 Left&20&0.8&no\\
			 Right&20&0.8&yes\\
			 Column Width&82&3.2&no\\
			 Column Spacing&6&0.25&no\\\hline
		\end{tabular}
	\caption{Spatial metadata needs and capabilities}
\label{tab:Margin_settings}
\end{table*}

\section*{ACKNOWLEDGEMENTS}\label{ACKNOWLEDGEMENTS}
This research is supported by National Science Foundation project BCS-2049837.

{
	\begin{spacing}{1.17}
		\normalsize
		\bibliography{foss4g_metadata_references} % Include your own bibliography (*.bib), style is given in isprs.cls
	\end{spacing}
}

\subsection{Identifying Software}\label{software}

We identified a short list of software to evaluate by searching for candidates on the FGDC's ISO Geospatial Metadata Editors registry ( https://www.fgdc.gov/metadata/iso-metadata-editor-review-v2), and OSGEO Projects (https://www.osgeo.org/projects/)


For example, we have excluded esri's robust metadata support , CatMDEdit is offically no longer maintained as of March 3, 2022. The most recent version (5.0) was released in 2014



USGS Metadata Wizard (v 2.0.7, updated March 1, 2022) implements FGDC-CSDGM (Federal Geographic Data Committee) (Content Standard for Digital Geospatial Metadata) metadata standards, allows customization of the default template, can validate records, can autopopulate spatial metadata and data dictionaries from data files, and allows batch processing with Python. Available at https://github.com/usgs/fort-pymdwizard with documentation at https://usgs.github.io/fort-pymdwizard/index.html

mdEditor web-based editor at https://www.mdeditor.org/ 

QGIS plugins include:
- QSphere (6.2.2, August 12, 2021, https://qgis.projets.developpement-durable.gouv.fr/projects/qsphere)
- MetaTools (0.3.1, March 18, 2016, https://github.com/nextgis/metatools)
- GeoCat Bridge - facilitates publishing maps from QGIS as web services, including metadata. However, GeoCat is provided as fee-for-service.

GeoServer - OGC services provider https://geoserver.org/ 
GeoNetwork - Catalog service
pycsw - Python catalog service library
GeoNode (3.2.0) Content management system uses pycsw and geoserver



The idea is to plan for oppenness from the start of the research project

Compendium / Research Project:
OSF has project-level metadata, abstract
GitHub has .cff citation file and main readme.md
"Making data 'interoperable' and 'reusable' can only be achieved if the data are annotated with comprehensive, standardized, high-quality metadata. Again, the absence of necessary metadata standards, appropriate ontologies, and easy-to-use annotation tools is a significant barrier." (137-138)






\section*{APPENDIX (Optional)}\label{APPENDIX}

Any additional supporting data may be appended, provided the paper does not exceed the limits given above. 


\end{document}
