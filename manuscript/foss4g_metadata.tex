% Version 2020-12-15
% update – 161114 by Ken Arroyo Ohori: made spacing closer to Word template throughout, put proper quotes everywhere, removed spacing that could cause labels to be wrong, added non-breaking and inter-sentence spacing where applicable, removed explicit newlines
% update – 010819 by Dennis Wittich: made spacing and font size closer to Word template, updated references and refernces style
% update – 042319 by Dennis Wittich: font size of captions set to 'small', first author names are shortened, hyphenation fixed
% update – 010620 by Dennis Wittich: Footnotes alignment set to left
% update - 151220 by Clement Mallet: Template adapted for double blind full paper submissions
% update - 060321 by Christian Heipke: Template refined for double blind full paper submissions
% update - 090921 by Christian Heipke: Template refined for double blind full paper submissions

\documentclass{isprs} % isprs class modified 23-04-2019 (Dennis Wittich)
\usepackage{subfigure}
\usepackage{setspace}
\usepackage{geometry} % added 27-02-2014 Markus Englich
\usepackage{epstopdf}
\usepackage[labelsep=period]{caption}  % added 14-04-2016 Markus Englich - Recommendation by Sebastian Brocks
\usepackage[british]{babel} 
\usepackage[hang]{footmisc}
\usepackage[authoryear,round,longnamesfirst]{natbib}
\def\footnotemargin{1em} % added 08-01-2020 Dennis Wittich

%\usepackage[authoryear]{natbib}
%\def\bibhang{0pt}

\geometry{a4paper, top=25mm, left=20mm, right=20mm, bottom=25mm, headsep=10mm, footskip=12mm} % added 27-02-2014 Markus Englich
%\usepackage{enumitem}

%\usepackage{isprs}
%\usepackage[perpage,para,symbol*]{footmisc}

%\renewcommand*{\thefootnote}{\fnsymbol{footnote}}
\captionsetup{justification=centering,font=normal} % thanks to Niclas Borlin 05-05-2016
\captionsetup[figure]{font=small} % added 23-04-2019 Dennis Wittich
\captionsetup[table]{font=small} % added 23-04-2019 Dennis Wittich

\begin{document}

\title{Mainstreaming metadata into research workflows to advance reproducibility and open geographic information science}
\date{}


% KAO: Remove extra spacing
% Anonymous submissions, authors' names should not be visible
\author{Joseph Holler\textsuperscript{1} and Peter Kedron\textsuperscript{2}}

% KAO: Remove extra newline
% Anonymous submissions, authors' affiliations should not be visible
\address{\textsuperscript{1}Department of Geography, Middlebury College - josephh@middlebury.edu \\
\textsuperscript{2}School of Geographical Sciences and Urban Planning, Arizona State University - peter.kedron@asu.edu}


% If the corresponding author is NOT the final author, always add a % space before the subsequent comma, i.e.
% first author name\textsuperscript{a,}\thanks{Corresponding author} , % second author name \textsuperscript{b}, etc.
% thanks to Niclas Borlin 05-05-2016


\commission{}{} %This field is optional. If filled, XX and YY should be replaced by adequate numbers. See https://www2.isprs.org/commissions/
\workinggroup{} %This field is optional.
\icwg{}   %This field is optional.

% KAO: Use times symbol
\abstract{
EDIT DOWN TO 200 WORDS
In this three-part research paper, we focus on metadata in research compendia and related research products through all phases of the research workflow.
First, we specify ideal requirements of geographic metadata in support of reproducible research workflows and open science.
We consider metadata needs at each step of the research process, including proposal writing, pre-analysis registration, ethics review board approval, data collection and analysis, publishing, and reproducing published research.
The metadata software needs assessment is based on literature review of reproducibility and open science, and on teaching and practising reproducibility with geographic methods.

Second, we review the Dublin Core and International Organisation for Standardization (ISO) geographic metadata standards and popular open source platforms for geospatial research and their support for the requirements of geographic metadata articulated in the first part.
The scope of the review includes metadata functionality available through spatial analysis software platforms, including R, Python, QGIS, GRASS and SAGA; and it also includes metadata or cataloguing tools, including GeoNetwork, GeoNode, the USGS Metadata Wizard, and mdeditor.

Finally, we articulate a vision for open source geographic metadata software development in support of open and reproducible human-environment and geographical research.
In this vision, metadata software tools shall integrate with executable research compendium to assist researchers with their workflow from inception to publishing and archiving.
The vision builds off our HEGSRR project, in which we independently reproduce and replicate published studies with open source geographic software, integrate reproduction and replication studies into project-based geographic information science courses, and develop curricula and infrastructure for reproducible research.
Each section of the paper is thus supplemented with experiences and examples drawn from the HEGSRR project.
Of particular relevance, we have already completed seven reproduction or replication studies with graduate and undergraduate students using open source geospatial software, encountering numerous barriers caused by inadequate use or documentation of metadata in research planning, execution, and archiving.
We have also developed a template Git repository compendium for reproducible research and prototyped its use in our studies and teaching, discovering software barriers to documenting metadata and opportunities to integrate metadata into more efficient and reproducible research practices.}

\keywords{Metadata, Open Science, Reproducibility, Open Source GIS}

\maketitle

%\saythanks % added 28-02-2014 Markus Englich

% KAO: Sloppy spacing ensures non-overfull lines. Can be removed if this is not an issue.
\sloppy

\section{Introduction}\label{Introduction}

Can the scientific community expand knowledge production and the scope of inquiry, accelerate and improve scientific communication, and improve scientific rigour?
These are the motivations and designs of open science principles---to expand the availability and usability of scientific research publications, data, and methods \citep{NASEM2018}, thereby making scientific studies more reproducible  and enabling new forms of inquiry based on synthesis and meta-analysis \citep{NASEM2018,NASEM2019}.
However, one key barrier to reproducible open science is a lack of standardised metadata documentation for research projects and associated data, code, and processing environments (Ibid.).
The National Academies of Sciences, Engineering and Medicine (NASEM) \textit{Open Science by Design} \citeyear{NASEM2018} report therefore calls for adherence to the FAIR principles \citep{Wilkinson2016} for sharing research data in a findable, accessible, interoperable, and reusable manner.
According to the report, achieving FAIR principles for open science will require infrastructure---data repositories and easy-to-use metadata annotation software---and additional labour on the part of researchers to document metadata in final the preservation phase of the research life cycle \citep{NASEM2018}. 
But, what if metadata inspired by FAIR principles have the potential to resolve  the reproducibility crisis in computational research, as \citep{Leipzig2021} suggest?

We propose that researchers should formally create metadata from the beginning of the research life cycle and maintain metadata throughout the full research work cycle.
Open source geographic information systems software should support metadata-rich research life cycles from inception through dissemination and preservation.

Our proposal is based on reproducibility literature in the social and geographical sciences and on our experience conducting reproductions and replications of geographic research and training geography students in reproducible research practices \citep{Kedron_Holler_Bardin_Hilgendorf_2022}.
In particular, we have developed templates for pre-analysis plan registrations, reproduction and replication study reports, and reproducible research compendiums \citep{Kedron_Holler_2022}, and we have applied the templates to seven reproduction and replication studies conducted with teams of students and and research assistants.
We suggest that rather than repeatedly and redundantly writing about their data through all phases of a research work cycle, researchers could formalise metadata at the project inception, use the metadata for much of the documentation required throughout the research process, and use software to maintain and track changes in metadata throughout the research process.
Furthermore, we anticipate that such a metadata-rich research life cycle would further enhance the quality and transparency of metadata in open science by more thoroughly and consistently recording information about data provenance, license, access, and distribution, and by supplanting for full access to restricted data.

In the following section, we introduce the open science research life cycles while reviewing the role of metadata in each phase of research.
We also review the most important metadata standards for documenting geographic research.
We then describe our methods for selecting and reviewing open source software tools for annotating and maintaining metadata in support of metadata-rich research life cycles.
We discuss the existing capabilities of open source software and conclude with suggested directions for future development in support of reproducible open science.

\section{Geographic Metadata for Reproducible Open Science}\label{sec:Background}

Open science aims to enhance the transparency, accessibility, and reproducibility of scientific research \citep{NASEM2018}.
In geographic information science, this can be achieved with open public domain data, open source GIS software, public research workflows, and peer review inclusive of data and workflows to the greatest extent permissible by data and software needs \citep{Singleton2016}. 
Following the National Academies of Science, Engineering and Medicine \citep{NASEM2019}, a reproduction study aims find the same results using the same data and methodology as a published study.
Once a study is reproducible, it becomes possible to reanalyse the original study design in a reanalysis by purposefully altering parameters or procedures \citep{Christensen2019}. 
A replication study aims to test the findings of a published study by collecting new data and following a similar methodology \citep{NASEM2019}.
Together, reproduction, reanalysis, and replication studies offer a deep understanding of the original research, test its credibility, and enhance the self-corrective mechanisms of the scientific community.
Over a series of replication studies, alternative hypotheses can be tested across geographic contexts to develop generalizable theories through the accumulation of evidence \citep{Kedron_Holler_2022}.

Open and reproducible science requires data to be findable, accessible, interoperable, and reusable (FAIR), and each of the four FAIR guiding principles require metadata \citep{Wilkinson2016, NASEM2018}.
Metadata is information describing data, providing essential context about the data so that other users can find, access, and use the data appropriately.
\citet{Kim1999} summarises seven essential components for geographic metadata:

%\itemize
\begin{enumerate}
\setlength\itemsep{0em}\setlength\parskip{0em}\setlength\topsep{0em}\setlength\partopsep{0em}\setlength\parsep{0em} 
\item{Identification, enabling cataloguing and search} 
\item{Data quality, informing fitness for use}
\item{Spatial data organisation, describing the data model and spatial support}
\item{Spatial reference, describing the coordinate system and datum}
\itme{Entity and attributes information, also known as a data dictionary}
\item{Distribution, including license, data format, and access procedures}
\item{Metadata reference about the author and date of the metadata itself}
\end{enumerate}

\citet{Schuurman2006} argue that these standardised metadata categories lack essential ontological context for data, including sampling method, attribute name definitions, measurement specifications, classification systems, data models, collection rationales, policy and legal context, measurement schemes, sampling, policy and legal constraints, and anecdotal information relevant for data use.
More complete knowledge of data sources requires ontological and perhaps even ethnographic study of the social context in which data was created \citep{Schuurman2008}.
In the context of metadata for open science and reproducibility, the extant international standards suffice for most purposes, while the full research publication and compendium associated with data are sufficiently flexible and detailed to provide the data's social and ontological context.
However, according to \citet[137-8]{NASEM2018}, ``Making data `interoperable' and 'reusable' can only be achieved if the data are annotated with comprehensive, standardized, high-quality metadata. Again, the absence of necessary metadata standards, appropriate ontologies, and easy-to-use annotation tools is a significant barrier.''
Although geography does have metadata standards (see section \ref{sec:Metadata}), open source geospatial software platforms may need improved tools necessary for mainstreaming geographic metadata into the full research life cycle.

\subsection{Open Science Research Life Cycle}\label{sec:lifecycle}

The 2018 \citeauthor{NASEM2018} report \textit{Open Science by Design} envisions open science practices in all six phases of the research workflow: provocation, ideation, knowledge generation, validation, dissemination, and preservation, while singling out the preservation phase for metadata documentation.

In the \textbf{provocation} phase, researchers review literature and data to identify opportunities for novel contributions.
In this phase, researchers benefit more from a metadata-rich open science framework than they contribute.
With open geographic data and metadata for published literature, researchers could design geographically-explicit bibliometric analyses and synthesis studies.
This is not currently possible in human-environment geography as \citet{Margulies2016} demonstrate in a review of 437 global change science case studies, for which the geographic extents of each case study were persistently ambiguous. 
Researchers would also have more detailed insight into the data and methodology of the studies they review with human-readable forms of metadata for each component of the study, helping to clarify the ambiguity of communicating complex computational methods with narrative publications.

In the \textbf{ideation} phase, researchers investigate data, and plan, prototype, and preregister research designs.
Three different types of plans are required of ethical and open research in this phase: 1) protocols for research with human subjects for ethical review, 2) research proposals and their associated data management plans (DMPs), and 3) pre-registered analysis plans (see table \ref{tab:Ideation_Info}).

In ethical research practices, researchers must propose their research protocol prior to collecting or analysing data, describe the analytical purpose of collecting the data, and explain how the data will be stored, protected, de-identified, and disseminated \citep{DHEW1978}.
Data use agreements for secondary data sources are also be required.

In research proposals and DMPs, researchers must describe how data will allow them to test hypotheses by measuring concepts therein, and describe how the data will be stored and disseminated \citep{NSF2021}.
Research funding agencies increasingly expect data management plans to explain how data will be made available to the public according to open science principles. 

In order to maximise replicability and inferential power, the ideation phase should include registration of a research plan (pre-registration) prior to data collection, requiring researchers to fully specify metadata for all of the research data and analyses that they intend to create \citep{Nosek2018}.
In deductive open science, researchers should strive to conduct objective unbiased studies by avoiding viewing or manipulating research data prior to registering their analysis plan \cite{Nosek2018}.
The plan should contain complete descriptions of all data, planned data transformations, and final analyses and outputs.
In order to accomplish this goal, secondary data sources should have sufficiently documented metadata with which researchers can plan their study without inspecting the data values directly.

In sum, the ideation phase requires researchers to study metadata for any secondary data sources they plan to use, and to specify metadata for any data they plan to create.
In the current state of practice, this metadata documentation is required in narrative form in a variety of documents.
We propose that formalising metadata documentation in this research phase could both mitigate redundancies and enhance transparency.

\begin{table*}[h]
	\centering
		\begin{tabular}{|c|c|c|c|c|}\hline
		   Requirement&Pre-registration&Proposal&DMPs&Human Subjects Protocol\\\hline
		     Author / PI                            & yes & yes & - & yes \\
		     Other Personnel                        & yes & yes & - & yes \\
    		 Title                                  & yes & yes & - & yes \\
    		 Abstract / Summary                     & yes & yes & - & yes \\
    		 Temporal Extent                        & yes & yes & - & yes \\
    		 Spatial Extent                         & yes & yes & - & yes \\
    		 Hypotheses                             & yes & yes & - & yes \\
    		 Sampling \& Recruitment Protocol        & yes & yes & - & yes \\
    		 Secondary data details                 & yes & yes & - & yes \\
    		 Secondary data access \& restrictions   & yes & yes & - & yes \\
    		 Survey instrument                      & - & - & - & yes \\
    		 Data collection methods                & yes & yes & - & yes \\
    		 Data variables                         & yes & yes & yes & yes \\
    		 Data analysis methods                  & yes & - & - & - \\
    		 Identifiable private information       & - & - & yes & yes \\
    		 Anonymity \& Confidentiality            & - & yes & yes & yes \\
    		 Data storage \& security during research& - & yes & yes & yes \\
    		 Data archiving \& dissemination         & yes & yes & yes & yes \\\hline
		\end{tabular}
	\caption{Ideation phase information requirements related to project and geographic metadata.}
\label{tab:Ideation_Info}
\end{table*}

In the \textbf{knowledge generation} phase, researchers use open source software tools to collect and analyse data in inter-operable formats with sufficient documentation, metadata, and computational notebooks to enable future reuse and replication.

Ideally, researchers will organise their materials and methods for computational research in a structured executable research compendium \citep{Singleton2016,Nust2021} containing all of the narrative, data, code, software, and computer scripts required to compile the final publication starting with raw data.
Computational notebooks like Jupyter notebooks or R Markdown are commonly used to interweave narrative with code in executable compendia (ibid).
It is recommended to store compendiums in version tracking systems like Git in order to preserve a full history of changes to the research project \citep{Stodden2014}.
Ideally, compendiums should implement a routine structure for research components, including directories for procedures or code, documents, raw data, processed data, and results \cite{Kedron_Holler_2022,Christensen2019,Marwick2018}. In addition to this structure, the compendium components should be well-documented with metadata \cite{Kedron_Holler_2022,Marwick2018}.


During the knowledge generation phase, metadata records should be maintained and updated with complete provenance information on the origins of the data and a history of all data transformations \citep{NASEM2019,Tullis2021}.
Provenance is essential for understanding the quality of data within the research and the quality and context of the research data if its to be reused elsewhere \citep{Tullis2021,Schuurman2006}.
The complexity of computational research in geography implies that provenance metadata is a necessary precursor for communication and reproduction of research methodologies.
Therefore, \citet{Anselin2014}, created a metadata system for spatial weights matrices in spatial statistics in a form that both records human-readable provenance and machine-readable instructions for reproducing the analysis.
As such complex computational research methods diverge from the original pre-registered plan, Git can track changes to metadata updated during the knowledge generation phase, lending transparency to intended changes and unintended deviations.

In the \textbf{validation} phase, researchers analyse, visualise, interpret, and validate results while sharing preliminary findings in working papers and conferences.
At this phase, the overall project and any public project component can be registered and assigned a persistent link like the digital object identifier (DOI) through digital repositories like Open Science Framework (OSF) or figshare.
Registration requires project-level metadata to enable archiving and searching.
Although researchers may be reluctant or constrained to release complete data at this phase, metadata can be shared for project components that must remain private or embargoed.

In the \textbf{dissemination} phase, the research is peer reviewed, revised, and published, ideally with associated data and code.
A version of the research compendium should be made available for the peer review process \citep{Singleton2016}, complete with research data, procedures, and metadata.
In response to reviews, any changes to the research procedures can ideally be documented and tracked through changes in metadata and release of a final version of the compendium.

Finally, in the \textbf{preservation} phase, the manuscripts, data, and code are placed in FAIR digital archives with final revisions of metadata.
In other words, an open access research compendium should be published and archived with metadata specifying access, licenses, data quality, and limitations. 
\citep{Wilson2021} propose a five-star system for rating the reproducibility of such compendia.
Publishing data and code with some metadata is only two-star level reproducibility: complete implementation of international metadata and encoding standards earns four stars.
For example, researchers could earn four stars by storing data with Open Geospatial Consortium (OGC) standard formats, document metadata with ISO standards, and encode the metadata in XML.
Documenting and containerizing the processing environment earns five.

Due to the proprietary, private, or voluminous nature of some data, it may not be possible to release a fully reproducible research compendium.
Researchers may need alter or fabricate alternative data for the purposes of reproducibility, e.g. through simulation, jittering, or sampling \citet{Tullis2021,Singleton2016}. 
In this case, metadata is essential for documentation of original data, means for accessing original data, and methods for creating alternative demonstration data.

In order to maximize the findability and legibility of the research compendium for both humans and machines, the overall repository and each of its components must be meticulously documented with metadata according to international standards \citep{Wilkinson2016,Wilson2021}.
For instance, GitHub repositories have readme and citation files to facilitate this, while OSF projects have project-level metadata and the ability to register generate DOI persistent identifiers.


 
\subsection{Metadata Standards for Geographic Research}\label{sec:Metadata}

Many disciplines lack sufficient standards for data sharing, interoperability, and documentation \citep{NASEM2019}.
The geographic sciences have standard formats, protocols, and algorithms for storing, distributing, and analysing geographic data---all coordinated by the Open Geospatial Consortium (OGC).
However, the OGC has left metadata standards to the spatial data infrastructures (SDIs) of individual states and regions, including the Federal Geographic Data Committee's (FGDC) Content Standard for Digital geographic metadata (CSDGM) in the United States, and the Infrastructure for Spatial Information in Europe (INSPIRE) \cite{Kim1999,Bartha2011}
Individual SDIs are increasingly following and harmonising with the International Standards Organization (ISO) series of geographic information metadata standards, especially the 19115 standard for geographic information metadata \citep{ISO2014} and the 19139 standard for encoding metadata in XML \citep{ISO2019}.
These standards are appropriate for individual geographic data layers.
While the ISO standards are copyrighted and costly to purchase, researchers may access them through the many SDIs and open source geospatial software projects that have implemented them.

At the project level, research publications and compendia should ideally be documented with the Dublin Core\texttrademark{}  standard metadata elements. Relevant elements include Title, Subject keywords, description (abstract), Type, Coverage (geographic and temporal), Creator, Publisher, Contributor(s), Rights, Date, 

%\itemize
\begin{enumerate}
\setlength\itemsep{0em}\setlength\parskip{0em}\setlength\topsep{0em}\setlength\partopsep{0em}\setlength\parsep{0em} 
\item{Title of the dataset} 
\item{Abstract (description)}
\item{Theme Keywords, including ISO19115 Topic}
\item{Place Keywords, look up more on this}
\itme{Purpose (context of data creation)}
\item{Data author}
\item{Association with larger works}
\item{Publication date}
\item{Link / identifier, e.g. DOI}
\item{Access constraints}
\item{Use constraints}
\item{Dataset credits}
\item{Time period represented in dataset}
\item{Native data set environment (processing environment, file name, file size)}
\item{Data Quality: Attribute accuracy, Logical Consistency, Completeness, Horizontal and Vertical Positional Accuracy}
\item{Process Steps: method, date, inputs, and outputs}
\item{Source inputs, including fairly complete reference information for each data source}
\item{Spatial support: description, bounding box, coordinate reference system, vertical datum, spatial data type, feature count}
\item{Entity and attribute: this is the data dictionary}
\item{Distribution: contact information, access instructions, liability disclaimer, data fees}
\item{Metadata contact}
\end{enumerate}

\section{Materials and Methods}\label{methods}

In this section, we describe our approach for reviewing metadata capabilities of open source geographic information software.

\subsection{Open source geographic information systems}\label{software}

Following \citet{Singleton2016}, we focus on open source software for the purposes of open science and reproducibility.
We identified a short list of software to evaluate by searching for candidates on the FGDC's ISO geographic metadata Editors registry (https://www.fgdc.gov/metadata/iso-metadata-editor-review-v2), and the OSGEO Projects (https://www.osgeo.org/projects/)


For example, we have excluded esri's robust metadata support , CatMDEdit is offically no longer maintained as of March 3, 2022. The most recent version (5.0) was released in 2014



USGS Metadata Wizard (v 2.0.7, updated March 1, 2022) implements FGDC-CSDGM (Federal Geographic Data Committee) (Content Standard for Digital geographic metadata) metadata standards, allows customization of the default template, can validate records, can autopopulate geographic metadata and data dictionaries from data files, and allows batch processing with Python. Available at https://github.com/usgs/fort-pymdwizard with documentation at https://usgs.github.io/fort-pymdwizard/index.html

mdEditor web-based editor at https://www.mdeditor.org/ 

QGIS plugins include:
- QSphere (6.2.2, August 12, 2021, https://qgis.projets.developpement-durable.gouv.fr/projects/qsphere)
- MetaTools (0.3.1, March 18, 2016, https://github.com/nextgis/metatools)
- GeoCat Bridge - facilitates publishing maps from QGIS as web services, including metadata. However, GeoCat is provided as fee-for-service.
- PgMetadata (1.1.1, February 14, 2022) writes metadata to a PostGIS database and automatically derives spatial metadata from PostGIS, using fields compatible with ckan and Data Catalog Vocabulary (DCAT)

GeoServer - OGC services provider https://geoserver.org/ 
GeoNetwork - Catalog service
pycsw - Python catalog service library
GeoNode (3.2.0) Content management system uses pycsw and geoserver


\subsection{What do we need from metadata software?}\label{metadataneeds}

Based upon our review in section \ref{lifecycle} above, we have enumerated useful characteristics for open-source software in support of open and reproducible research (see table \ref{tab:Metadata_Software}).
The specific characteristics fall into three main categories: (1) ease of use and start-up, (2) implementation of metadata standards, and (3) automated features to facilitate metadata management.

First, Metadata software should be easy to set up and to use in order to ease the burden of metadata documentation and management on precious research resources.
Software support for metadata management varies tremendously with regard to set-up and installation.
For example, content management systems (CMS) require installation, server administration, and user login prior to working with any metadata. 
Plugins for desktop GIS software or packages for computer languages require installation and familiarity with a particular  processing environment prior to installing and using the metadata software. 
Standalone desktop metadata editors tend to be very easy to install and start using straight away, while internet-based metadata editing services require only a web browser and login.

Graphical user interfaces (GUIs) enhance ease of use and learning how to document metadata, especially for novice users.
Interactive features can aid users with features like help documentation about each metadata field, auto-populated lists of keywords from standard dictionaries, selection of spatial and temporal extents, highlighting incomplete required fields, and organising complex information into separate sections or tabs.

Second, in an open science framework, geographic metadata should be documented with common standards. Dublin Core is the ideal standard for documenting the overall research project. ISO 19115 is ideal international standard for documenting individual data layers.
However, the FGDC CSDGM standard for data layers is very similar to the ISO 19115 standard and has been widely used by the federal government.
If the software does not support these common metadata standards, then the metadata may prove useful internally, but it will not easily be integrated with archives or CMSs or included in automated synthesis or meta-analysis research.

Once metadata conforms to international standards, it should also be encoded and stored with open machine- and human-readable standard formats. 
The use of an open standard and open machine-readable format enables interoperability with CMSs and automated synthesis research algorithms.
If metadata is readable by computers with common parsers, then the metadata can be integrated with more general research management tools to perform functions like submitting and updating pre-analysis plans, data management plans, or human subjects research protocols, or for generating and updating documentation for a research compendium.
In addition, Git can track versions of text-based formats like Extensible Markup Language (XML), JavaScript Object Notation (JSON), and YAML Ain't Markup Language (YAML).
This implies that as metadata changes over time, Git can be used to visualise differences in metadata over different phases of a research project, from pre-registration of the analysis plan to reporting results and finalising the peer review process.
Git's difference visualisation could highlight changes in spatial extent, the data dictionary of variables to be collected or computed, protocols for access, and more. 

Third, metadata software can---to a great extent---automate the discovery, creation, and verification of metadata.
In terms of discovery, software can catalogue data layers and partially automate documentation of geographic and attribute metadata.
Many desktop GIS and geographic data catalogues can automatically catalogue the geographic data in a research compendium by parsing computer directories in search of recognised geographic data types, resulting in a list of any potential geographic data sources and their relative locations on a computer drive.
This feature is particularly useful for routinely cataloguing the data sources contained in a research compendium and verifying the completeness of metadata records for the compendium.
Software can also be programmed to automatically extract or calculate geographic metadata, including coordinate reference systems, data types, and spatial extents.
Attribute data can be extracted to facilitate creation and maintenance of data dictionaries, including variable names, attribute data types, feature or observation counts, descriptive statistics for quantitative data, and unique values for categorical data.

If software contains all the features for cataloguing geographic data and much of its geographic and attribute metadata, then it can also be extended to perform an audit of individual metadata records or the records for an entire research compendium.
This feature would crosscheck metadata documentation with all automatically derived metadata to report any irregularities or missing information.
In addition, an audit should check for compliance with regard to completeness of other required metadata fields which cannot be automatically derived, e.g. authorship, license, and distribution information.

None of this yet ensures compliance with one of the most important functions of metadata: to record provenance.
Researchers working exclusively with Python, R or other computing languages will hopefully have recorded a complete history of data transformations and manipulations in legible code, and this method of provenance documentation requires the metadata to link to permanently accessible code.
Another approach to provenance is to use analytical software that records each step of data transformation as metadata attached to the data itself, which can then be included in the formalised metadata record.
Depending on the software environment, this metadata may even be used as a script of instructions for the software to reproduce the data transformations.

Finally, and beyond the three categories mentioned above, some metadata software may be accessed with an application programming interface (API).
An API can allow other software programs to access the functions and data structures of the metadata software, facilitating its integration with other applications and development of more specialised research data management software.

If any geographic metadata software could implement all of the features described above, it would be easier for researchers to document their projects and data according to inter-operable and international standards, easier to control the completeness and accuracy of geographic metadata records, and easier to mobilise metadata to support the full research workflow, thereby reducing redundancy and increasing transparency.

\subsection{Software}

There are several types of open source software which may be useful for authoring and maintaining geographic metadata.
Desktop GIS like QGIS, GRASS, SAGA, and gvSIG are designed for interactive data visualization, editing, and processing.
The R and Python programming languages are increasingly used for geospatial analysis, prompting development of specialised packages for managing geographic metadata in those languages. Examples include the geometa package for R and the pygeometa package for Python.
Catalogue services like GeoNetwork and pycsw are designed for maintaining and sharing databases of searchable geographic metadata.
Content Management Systems (CMS) like GeoNode and ckan are designed to store and share geographic data in searchable web-accessible archives.
Finally, specialised metadata authoring software like Metadata Wizard and mdEditor allow users to author and maintain geographic metadata in a stand-alone application.

\begin{table*}[h]
	\centering
		\begin{tabular}{|c|c|c|c|c|}\hline
		    Requirement&QGIS&GeoNode&GeoNetwork\\\hline
			 Easy start-up & - & - & - \\
			 ISO & - & - & - \\
			 FGDC CSDGM & - & - & - \\
			 Dublin Core & - & - & - \\
			 Automate Data Catalogue & - & - & - \\
			 Automate Geographic Metadata & - & - & - \\
			 Automate Attribute Metadata & - & - & - \\
			 Audit Metadata Record(s) & - & - & - \\
			 Track Provenance & - & - & - \\
 			 GUI Editor & - & - & - \\
 			 Re-usable Content & - & - & - \\
			 Encoding (XML, JSON, and/or YAML) & - & - & - \\
			 API & - & - & - \\\hline
		\end{tabular}
	\caption{Spatial metadata needs and capabilities.}
\label{tab:Metadata_Software}
\end{table*}

\section*{ACKNOWLEDGEMENTS}\label{ACKNOWLEDGEMENTS}
This research is supported by National Science Foundation project BCS-2049837.

{
	\begin{spacing}{1.17}
		\normalsize
		\bibliography{foss4g_metadata_references} % Include your own bibliography (*.bib), style is given in isprs.cls
	\end{spacing}
}

\section*{APPENDIX (Optional)}\label{APPENDIX}

Any additional supporting data may be appended, provided the paper does not exceed the limits given above. 


\end{document}
